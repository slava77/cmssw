\input epsf

\parindent0pt

\def\file#1{{\bf #1}}
\def\DescriptionBegin{}
\def\DescriptionEnd{\vskip\baselineskip}

\epsfxsize=\hsize
\epsfbox{schema.eps}
\vskip\baselineskip

\obeylines
\file{TotemRPGeometryBuilder/interface/DetGeomDesc.h}
\DescriptionBegin
Class resembling GeometricDet class. Slight changes were made to suit needs of the TOTEM RP description.
Each instance is a tree node, with geometrical information from DDD (shift, rotation, material, ...), ID and list of children nodes.
It is intended to have two such a trees. One for ideal geometry (within IdealGeometryRecord) and second for real geometry (RealGeometryRecord).
The transition from ideal to real geometry (i.e. loading alignments) is done by TotemRPRealGeometryModule.
\DescriptionEnd

\file{TotemRPGeometryBuilder/interface/DDDTotemRPConstruction.h}
\DescriptionBegin
Class to build scructure of DetGeomDesc objects out of DDCompactView (resp. DDFilteredView).
It adds detector IDs (via class TotemRPDetId).
intended to be called from: modul TotemRPDetGeomDescESModule.
\DescriptionEnd

\file{TotemRPGeometryBuilder/interface/GeometryTestModule.h}
\DescriptionBegin
Testing module.
\DescriptionEnd

\file{TotemRPGeometryBuilder/interface/MisalignmentESModule.h}
\DescriptionBegin
A module to produce misalignments (in the form of Alignments class) in the RealGeometryRecord.
It can be used to produce differences between ideal and real geometry.
\DescriptionEnd

\file{TotemRPGeometryBuilder/interface/TotemRPAligner.h}
\DescriptionBegin
This class will be used to perform alignment in the future.
It is EDLooper, i.e. it makes several loops over all events.
At each loop it adjusts the real geometry (i.e. structure of DetGeomDesc in RealGeometryRecord).
\DescriptionEnd

\file{TotemRPGeometryBuilder/interface/DDDTotemRPCommon.h}
\DescriptionBegin
Some DDD related definitions, shared by different classes
\DescriptionEnd

\file{TotemRPGeometryBuilder/interface/TotemRPGeometry.h}
\DescriptionBegin
This is kind of "public relation class" for the tree structure of DetGeomDesc. It provides convenient interface to
answer frequently asked questions about the geometry of TOTEM Roman Pots. These questions are of type:
a) If detector ID is xxx, what is the ID of corresponding station?
b) What is the geometry (shift, roatation, material, etc.) of detector with id xxx?
c) If RP ID is xxx, which are the detector IDs inside this pot?
d) If hit position in local detector coordinate system is xxx, what is the hit position in global c.s.?
etc. (see the comments in definition bellow)
This class is built for both ideal and real geometry. I.e. it is produced by TotemRPIdealGeometryESModule in 
IdealGeometryRecord and similarly for the real geometry
\DescriptionEnd

\file{TotemRPGeometryBuilder/interface/TotemRPIdealGeometryESModule.h}
\DescriptionBegin
It converts DDCompactView to tree of DetGeomDesc nodes. The DDCompactView is created from DDL files by XMLIdealGeometryESSource.
It exploits DDDTotemRPContruction class.
Having built the DetGeomDesc tree, it creates TotemRPGeometry object.
\DescriptionEnd

\file{TotemRPGeometryBuilder/interface/TotemRPRealGeometryESModule.h}
\DescriptionBegin
This class copies tree of DetGeomDesc from IdealGeometryRecord and applies alignment corrections (if there is an Alignments class in
RealGeometryRecord). The result is saved in RealGeometryRecord.
Having built the DetGeomDesc tree, it creates TotemRPGeometry object.
\DescriptionEnd

\file{TotemRPGeometryBuilder/interface/TotemRPExtractAlignments.h}
\DescriptionBegin
This class compares two trees of DetGeomDesc nodes. It is done via itermediate class TotemRPGeometry.
It extracts all detectors from both structures and compare their positions and rotations. If they
differ, it adds an entry to the Alignments vetor. Hence, there is sort of zero-suppression.
\DescriptionEnd

\end
